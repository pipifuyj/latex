\documentclass[10pt,letterpaper]{article}

\usepackage{hyperref}
\usepackage{geometry}

% Fonts
%\usepackage[T1]{fontenc}
%\usepackage[osf]{mathpazo}
\usepackage{color}

% Set your name here
\def\name{Yanjie Fu}

% The following metadata will show up in the PDF properties
\hypersetup{
  colorlinks = true,
  urlcolor = black,
  pdfauthor = {\name},
  pdfkeywords = {computer science, data mining,PhD},
  pdftitle = {\name: Curriculum Vitae},
  pdfsubject = {Curriculum Vitae},
  pdfpagemode = UseNone
}

%\geometry{
%  body={6.0in, 8.5in},
%  left=1.0in,
%  top=1.0in
%}

\geometry{textheight=8.5in, textwidth=6in}

% Customize page headers
\pagestyle{myheadings}
\markright{\name}
\thispagestyle{empty}

% Custom section fonts
\usepackage{sectsty}
%\sectionfont{\rmfamily\mdseries\Large}
\subsectionfont{\rmfamily\mdseries\itshape\large}

% Other possible font commands include:
% \ttfamily for teletype,
% \sffamily for sans serif,
% \bfseries for bold,
% \scshape for small caps,
% \normalsize, \large, \Large, \LARGE sizes.

% Don't indent paragraphs.
\setlength\parindent{0em}

% Make lists without bullets and compact spacing
\renewenvironment{itemize}{
  \begin{list}{}{
    \setlength{\leftmargin}{1.5em}
    \setlength{\itemsep}{0.25em}
    \setlength{\parskip}{0pt}
    \setlength{\parsep}{0.25em}
  }
}{
  \end{list}
}

\begin{document}

% Place name at left
{\huge \bf \name}

% Alternatively, print name centered and bold:
%\centerline{\huge \bf \name}

\vspace{0.25in}

\begin{minipage}[t]{0.55\textwidth}
  School of Information Science and Engineering\\
  Graduate University of Chinese Academy of Sciences\\
  \\
  Center for Space Science and Applied Research\\
  Chinese Academy of Sciences\\
\end{minipage}
\begin{minipage}[t]{0.45\textwidth}
  Office: +86-010-62586433\\
  Mobile: +86-18611199228\\
  Fax: +86-010-62586433\\
  Email: \href{mailto:fu_yanjie@cssar.ac.cn}{\texttt{fu\_yanjie@cssar.ac.cn}}\\
  Homepage: \href{http://sites.google.com/site/pipifuyj/}{\texttt{http://sites.google.com/site/pipifuyj/}} \\
\end{minipage}

\section*{Biography}
Yanjie Fu received his B.E. degree in Computer Science from University of Science and Technology of China. In 2008, he was recommended as an exam-free graduate student, and now is a M.S. candidate in School of Information Science and Engineering, Graduate University of Chinese Academy of Sciences.

\section*{Education}
\begin{itemize}
  \item Master(2008.09-2011.07)
    \begin{itemize}
    \item 
      School of Information Science and Engineering,\\
      Graduate University of Chinese Academy of Sciences (GUCAS).
%   	 \begin{itemize}
%		\item \textit{Overall GPA:} 
%		\item \textit{Ranking:} 
%    	 \end{itemize}
    \end{itemize}

  \item Bachelor(2004.09-2008.07)
    \begin{itemize}
    \item
      Department of Computer Science,\\
      University of Science and Technology of China (USTC).
%  	 \begin{itemize}
%		\item \textit{Overall GPA:} 
%		\item \textit{Ranking:}  
%   	 \end{itemize}
    \end{itemize}
\end{itemize}


\section*{Research Interests}
\begin{itemize}
  \item Data Mining;
  \item Data Grid, Large Scale Data Sharing Technologies, including their application on satellite data;
  \item Information System, Virtual Observatory.
\end{itemize}

\section*{Honors \& Awards}
\begin{itemize}
  \item IEEE WCCI  Active Learning Competition Winner On Hand Writing Recognition(Rank 1), 2010
  \item Outstanding Graduate Scholarship, University of Science and Technology of China, 2007-2008 
  \item Outstanding Graduate Scholarship, University of Science and Technology of China, 2006-2007
  \item Outstanding Student Scholarship, University of Science and Technology of China, 2005-2006
  \item Outstanding Freshman Scholarship, University of Science and Technology of China, 2004-2005
  \item Outstanding Student of USTC Summer Research Project, 2006/07-2006/09
\end{itemize}

\section*{English Certifications}
\begin{itemize}
  \item College English Test-Bank4: passed
  \item College English Test-Bank6: passed
  \item GRE: V:340/800 Q:800/800 AW:4/6
\end{itemize}

\section*{Academic Activities}
\begin{itemize}
  \item  2010.07 The 10th Symposium of Scientific Databases And Information Technology (Guizhou,China)
  \item  2010.07 NASA Data Management and Data Interoperability Workshop - Planetary Data System (Shandong,China)
  \item  2010.05 AISTAT Active Learning Workshop (Sadinia,Italy)
	\begin{itemize}
		\item \textit{Presentation:} Clustering and Association in Active Learning
	\end{itemize}
  \item  2009.11 Chinese Virtual Observatory Workshop (Chongqing,China)
	\begin{itemize}
		\item \textit{Presentation:} The Architecture Of Space Science Virtual Observatory
	\end{itemize}
  \item  2009.10 National Physics Conference (Beijing,China)
\end{itemize}

\section*{Publications}
\subsection*{Workshop Papers}
\begin{itemize}
\item (1) Yanjie Fu, Ziming Zou, Jizhou Tong, Architecture Research of Space Science Virtual Observatory Based On Autonomic Grid, The 10th Symposium of Scientific Databases And Information Technology (\href{http://sites.google.com/site/pipifuyj/documents/ArchitectureResearchofSpaceScienceVirtualObservatoryBasedOnAutonomicGrid.pdf?attredirects=0&d=1}{\underline {PDF}} in Chinese)
\end{itemize}

\section*{Projects}

\subsection*{Service Oriented Architecture and its applications on Web Service}

\begin{itemize}
\item  1. My Donor \href{https://admin.ustcif.org/mydonor/frontend/root.php/}{\underline{Demo}}
	\begin{itemize}
	\item Aims
		\begin{itemize}
		\item With the increasingly growing large scale data of alumni, on one hand, people need to manage detailed records for donors, prospects, volunteers, memberships, alumni and more. On the other hand, to raise more money, it is quite necessary and important to track alumni grants,provide online fundraising approach and propose capital campaigns. Finally and also the most important thing is how to dig out those potential kind-hearted alumni and keep in touch with them. So, that is why we need "my donor". 
		\end{itemize}
	\item Programming Languages and Tools
		\begin{itemize}
		\item PHP+XML+ExtJS+MYSQL+ Statistics Method
		\end{itemize}
	\item Sponsor
		\begin{itemize}
		\item USTC Initiative Foundation
		\end{itemize}
	\end{itemize}

\item  2. Ways to Give 
	\begin{itemize}
	\item It is not always a good way to provide only call-in donation. With the development of e-business, an integrated online donation front end is significant to provide easy ways for alumni to donate.  And it is a project supported by USTC Initiative Foundation.
	\item Sponsor
		\begin{itemize}
		\item USTC Initiative Foundation
		\end{itemize}
	\end{itemize}

\item  3. Lightweight Php Web Framework \href{http://github.com/pipifuyj/phpframework}{\underline{Source Code}}
	\begin{itemize}
	\item We are always using other Web Framework to create web service or online system. However, we never think about if we can also create a new lightweight Web Framework, even our framework is still not so strong. But, we can understand what is the core mechanism of a web framework. And it is supported by personal interests with my dear friends in Beijing.
	\item It is a Model-View-Controller framework. Our main idea of this framework is as follows:
	\begin{itemize}
		\item (1)class "framework" has models+controllers+views;
		\item (2)obtain the action name and method name according to the URL;
		\item (3)require the relative controller PHP file and create an object for this controller class;
		\item (4)require the relative view PHP file and create an object for this view class;
		\item (5)we fire the events according to a single-line order - "beforeInitiation,afterInitiation,beforeRender,afterRender,beforeOutput,afterOutput"
		\item (6)we view model as an object: data+actions. model has fields and records. Each record means a row in a data matrix. 
	\end{itemize}
	\end{itemize}
\end{itemize}

\subsection*{Data Mining}
\begin{itemize}
\item  1. \href{http://www.wcci2010.org/competition-program/}{IEEE WCCI 2010 Competition Program} - Data mining (IJCNN) on Hand Writing Recognition \href{http://github.com/pipifuyj/extweka}{\underline {Source Code}} \href{http%3A%2F%2Fgithub.com%2Fpipifuyj%2Factivelearning%2Fblob%2Fmaster%2Fpipifujie.doc}{\underline {Award}} \href{http%3A%2F%2Fclopinet.com%2Fisabelle%2FProjects%2FAISTATS2010%2FAISTATS_slides%2FFlyingsky_YanjieFu.ppt}{\underline {Technical Report}} \href{http://clopinet.com/isabelle/Projects/AISTATS2010/}{\underline {Workshop}}
	\begin{itemize}
	\item Challenge Protocol
		\begin{itemize}
		\item The participants are allotted a budget of "virtual cash" allowing them to "purchase" all the training data labels at the price of 1 ECU (experimental cash unit) per label. They can place queries to the server by providing a list of samples for which they desire to purchase the label. Upon receipt of the labels, their account of virtual cash is debited. The participants are free to choose the number of queries and the number of samples per query. An experiment terminates when all the budget is spent or the challenge deadline is reached. To monitor progress, the participants are asked to provide predictions for all the labels every time they place a query, including the known and unknown labels of the training examples and the labels of the test examples.
		\item When query, an predict and a sample should be done:
		\begin{itemize}
			\item Predict - Using the examples with known labels (at first use the seed example, which has a positive label), train a predictor and provide prediction scores for ALL the examples of the dataset (including those used for training). Any sort of numeric prediction score is allowed, larger numerical values indicating higher confidence in positive class membership.
			\item Sample - Choose among the remaining unlabeled examples those for which you want the purchase labels. You may only query examples in the first half of the dataset (training examples). If you query test examples, you will not receive those labels (and not be charged either).
		\end{itemize}
		\item The goal is to purchase as few labels as possible with "virtual cash" while getting as good performance as possible.
		\end{itemize}
	\item Data Set
		\begin{itemize}
		\item There are 20000 instances in which every instance have 12000 attributes and 1 unknown label what should be predicted. The possible values for the label are 1 and -1. At first, we know the first instance's label is 1.
		\end{itemize}
	\item Programming Language and Tools
		\begin{itemize}
		\item java+python+extweka+libsvm
		\end{itemize}
	\item Result
	\item Award
		\begin{itemize}
		\item Winner(Rank 1) on Hand Writing Recognition
		\end{itemize}	
	\end{itemize}
\item 2. Space Weather Prediction and Data Analysis for Satellite Orbit
	\begin{itemize}
	\item Aims
	\begin{itemize}
	\item In the past, physics scientists present their space weather models based on physics theory and use those models to predict space weather and space events. However, the increasingly growing records from satellites leads to a problem that data is over-loaded for scientists. So, how to process large scale satellite data? We should seek help from data mining technologies.
	\item Now, for example, we can divide our data into two kinds: A is data from all kinds of satellite monitors; B are records from historical space events. But A is related to B. And we can view relationship between A and B as C. Consequently, can we dig out the patterns from A+B+C by mathematical approaches? And then translate those patterns into predictors to predict potential space events for specific possibility? Maybe the results are not totally accurate. But they are meaningful.
	\item So this project aims to forecast space event and provide potential satellite-disaster warming for Air Force. The business flow is as follows:
	\begin{itemize}
		\item (1)Obtain real-time data from space weather monitoring satellite
		\item (2)Data preprocessing including: error correction, data dimension reduction by sampling frequency, intermediate physical variables computing
		\item (3)Based on historical space weather event records, forecast potential space  weather disaster by data mining method like SVM, Classification, Neural Network. ( What I mainly focus)
	\end{itemize}
	\end{itemize}

	\item Sponsor
		\begin{itemize}
		\item National 973 Plan
		\end{itemize}
	\end{itemize}

\item 3. Finding Topics in A Collection of Documents \href{http://github.com/pipifuyj/topicfinding}{\underline {Source Code}}  \href{http://sites.google.com/site/pipifuyj/documents/TopicFindingReport.pdf?att
irects=0}{\underline {Technical Report}}
	\begin{itemize}

	\item Aims
		\begin{itemize}
		\item In our daily life, every document has its topic- what this document is mainly talking about. And machine can tell the difference of  documents' topics from composition of words. For example, if most words in a document are relative with sports, it is quite possible its topic is sports. So, the question is how to find out the topics in a collection of documents without training dataset. Only by simple clustering? Of course not.
	 	\item In our work, we use two methods: Kmeans and Shared Nearest Neighbor, and then improve the classic Shared Nearest Neighbor Approach based on Google Page Rank, and compare the clustering accuracy with Kmeans and classic Shared Nearest Neighbor approach.
		\end{itemize}

	\item Programming Languages and Tools
		\begin{itemize}
		\item java
		\end{itemize}

	\item Result

	\end{itemize}

\end{itemize}

\subsection*{Database and Large Scale Data Sharing Technology}
\begin{itemize}
\item 1. Database and Large Scale Data Sharing Technology \href{https://ssdg.cssar.ac.cn/}{\underline{Demo}}
\begin{itemize}
	\item Sponsor
	\begin{itemize}
	\item Chinese National Eleventh Five-Year Plan
	\end{itemize}
	
	\item Programming Language and Tools
	\begin{itemize}
	\item PHP+ExtJS+Google API+MYSQL+JAVA+JSP
	\end{itemize}

	\item Resource Layer
	\begin{itemize}
	\item Space science data is quite different from other common data. Frequently, space science data is stored in CDF format or Plain Text format or FITS format. At the same time, space science data is always located in different stations, observatories and institutes. So, in some way, it is a distributed file system. But to search, locate and retrieve needed data as quickly as possible, we need a metadata model or a metadata standard. We can divide metadata into two kinds by its formats. One is metadata in file; another is metadata in database. 
Another problem is how to design metadata model including category, instrument, equipment, space region and so on. To make the metadata model properly suite the usage of space science community, we need to take a lot of factors into consideration. 
	\item So, a dataset should contain data, standard metadata and core metadata. Here, data support physics experiments and analysis; standard metadata means metadata in file format which describe the detailed information of data, and it supports interoperability; core metadata means metadata in relative database which only contains important core attributes, and it supports data discovery. Data provider is a larger concepts including many datasets. 
Resource Layer provide data access and storage.
	\end{itemize}

	\item Service Layer
	\begin{itemize}
	\item Resource Register and Delivery
	\item Resource Audit
	\item Resource Discovery
        \item Metadata Harvest and Synchronization
      	\item Data and Service redundancy
      	\item Transparent Switching
        \item Single-Sign On
	\item Work Flow For Satellite Data Preprocessing
	\end{itemize}
	
	\item Service Layer
	\begin{itemize}
	\item Space Weather Model Computing 
	\item Google Earth Visualization Service
      	\item Atmosphere Density and Temperature Models
	\item Space Weather Event Association Data Discovery
	\end{itemize}
\end{itemize}
\end{itemize}

\bigskip

% Footer
\begin{center}
  \begin{small}
    Last updated: \today
  \end{small}
\end{center}

\end{document}